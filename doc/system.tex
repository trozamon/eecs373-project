\documentclass{article}

\usepackage{fullpage}
\usepackage{graphicx}
\usepackage{mathtools}

\begin{document}

\title{Green$^2$ Thumb System Design}
\author{
    Jason Leung\\
    Christopher Oo\\
    Jacob Perrin\\
    Alec Ten Harmsel
}
\date{}
\maketitle

\section{Power}
% TODO Check the maths
Power is provided by a bank of capacitors, charged by a solar cell. For the
purposes of planning, the following characteristics are assumed:

\begin{itemize}
    \item 66mW Power Draw: Nordic MCU + BLE Combo (33mW) and miscellaneous
        other power draws
    \item 10uA solar cell current
    \item 10ms runtime
\end{itemize}

\subsection{Maths}

The charging time can be found by manipulating the standard capacitor equation
thingy: \begin{align*}
    i & = C\frac{dv}{dt}\\
    it & = Cv\\
    t & = \frac{Cv}{i}
\end{align*}

\subsubsection{Assuming 5V Storage}
Need a continuous 66mW power while capacitors discharge from 5V to 3.3V.
Relevant maths: \begin{align*}
    p & = iv\\
    v & = v_oe^{-t/RC}\\
    i & = \frac{p}{v}
\end{align*}

The capacitor bank can decay from 5V to 3.3V in 0.1s, giving a time constant
of: \begin{align*}
    3.3 & = 5e^{-t/RC}\\
    RC & = 0.24
\end{align*}

Our current is then: \begin{align*}
    \frac{0.066}{5e^{-t/0.24}} = \frac{1}{3}e^{t/0.24}
\end{align*} and the integral from 0 to 0.1 gives us the charge: \begin{align*}
    \frac{1}{3} \cdot 0.24 \cdot \left(e^{0.1/0.24} - e^{0}\right)
\end{align*} which, when dividing by 5V, gives 8mF of capacitance needed.

\end{document}
